\documentclass{article}

% Set paper size and margins.
\usepackage[a4paper]{geometry}

% Do not indent paragraphs and inert blank space between them.
\usepackage{parskip}

% For \mathbb etc.
\usepackage{amssymb}

% For \eqref etc.
\usepackage{amsmath}

% Link colors.
\usepackage[colorlinks,linkcolor=blue,urlcolor=blue,citecolor=blue]{hyperref}

% To generate references.
\usepackage{biblatex}

% Bibliography file.
\addbibresource{ref.bib}

% Do not display date.
\newcommand{\created}[1]{\newcommand{\createdate}{#1}}
\newcommand{\updated}[1]{\newcommand{\updatedate}{#1}}

% Set author.
\author{Susam Pal}

\title{Periodic Functions with Sum as Identity Function}
\created{30 Jan 2019}
\documentclass[11pt]{article}

% Set paper size.
\usepackage[a5paper]{geometry}

% Use links in table of contents.
\usepackage[colorlinks=true,linkcolor=blue,urlcolor=blue]{hyperref}

% Add vertical space between paragraphs.
\usepackage{parskip}

% Math symbols.
\usepackage{amsmath}

% Colors
\usepackage{xcolor}

% Code examples.
\usepackage{listings}

% To include CC-BY license logo.
\usepackage{graphicx}

% Do not use monospace font for URLs.
\urlstyle{same}

\lstdefinestyle{cc}{
    language=c++,
    basicstyle=\small\ttfamily\color{blue!50!black},
    commentstyle=\color{green!50!black},
    directivestyle=\color{teal!75!black},
    identifierstyle=\color{violet!75!black},
    keywordstyle=\color{blue},
    stringstyle=\color{magenta},
    columns=fullflexible,
    keepspaces=true,
    upquote=true,
    showstringspaces=false,
}

\author{Susam Pal}
\title{Project Euler Solutions}

\newcommand{\simpledate}{%
    \the\day{}
    \ifcase\month \or Jan \or Feb \or Mar \or Apr \or May \or Jun
                  \or Jul \or Aug \or Sep \or Oct \or Nov \or Dec \fi
    \the\year{}%
}

\newcommand{\updatedate}{\simpledate}

\date{
    \small Last updated on \updatedate

}

\newcommand{\ccfile}[2][]{\lstinputlisting[style=cc,#1]{#2}}
\newcommand{\add}[1]{
    \input{#1}
    \pagebreak
    \subsection{Program}
    \ccfile{../c++/#1.cc}
    \pagebreak
}

\setcounter{tocdepth}{1}

\begin{document}

\section*{Problem}
Find two periodic functions \( f \) and \( g \) from \( \mathbb{R} \)
to \( \mathbb{R} \) such that their sum \( f + g \) is the identity
function. The axiom of choice is allowed.

A function \( f \) is periodic if there exists \( p > 0 \) such that
\( f(x + p) = f(x) \) for all \( x \) in the domain.

\section*{Solution}
The axiom of choice is equivalent to the statement that every vector
space has a basis. Since the set of real numbers \( \mathbb{R} \) is a
vector space over the set of rational numbers \( \mathbb{Q} \), there
must be a basis \( \mathcal{H} \subseteq \mathbb{R} \) such that every
real number \( x \) can be written uniquely as a finite linear
combination of elements of \( \mathcal{H} \) with rational coefficients,
i.e.,

\[
    x = \sum_{a \in \mathcal{H}} x_a a
\]

where each \( x_a \in \mathbb{Q} \) and \( \{ a \in \mathcal{H} \mid x_a
\ne 0 \} \) is finite. The set \( \mathcal{H} \) is also known as the
Hamel basis.

We know that \( b_a = 0 \) for distinct \( a, b \in \mathcal{H} \)
because \( a \) and \( b \) are basis vectors. In the above expansion of
\( x \), each \( x_a \) is a rational number that appears as the
coefficient of the basis vector \( a \). Therefore \( (x + y)_{a} = x_a
+ y_a \) for all \( x, y \in R \). Thus \( (x + b)_{a} = x_a + b_a = x_a
+ 0 = x_a \). This shows that a function \( f(x) = x_a \) is a periodic
function with period \( b \) for any \( b \in \mathcal{H} \setminus
\{a\} \).

Let us define two functions:
\[
    f(x) = \sum_{a \in \mathcal{H} \setminus \{ b \}} x_a a,
    \hspace{1em}
    g(x) = x_b b.
\]

where \( b \in \mathcal{H} \) and \( x \in \mathbb{R} \). Let us
choose \( c \in \mathcal{H} \) such that \( c \ne b \). Then \( f(x) \)
is a periodic function with period \( b \) and \( g(x) \) is a periodic
function with period \( c \). Further,

\[
f(x) + g(x)
= \left( \sum_{a \in \mathcal{H} \setminus \{ b \}} x_a a \right) + x_b b
= \sum_{a \in \mathcal{H}} x_a a
= x.
\]

Thus \( f(x) \) and \( g(x) \) are two periodic functions such that
their sum is the identity function.

\nocite{pfml-periodic-functions}
\nocite{radcliffe-periodic-functions}
\nocite{youcis-dim-r-q}
\printbibliography
\end{document}

