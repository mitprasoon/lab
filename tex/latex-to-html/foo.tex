\documentclass{article}
\usepackage{parskip}
\usepackage{amsmath}
\usepackage{amsthm}
\usepackage[colorlinks=true,linkcolor=blue,urlcolor=blue]{hyperref}
\newtheorem{theorem}{Theorem}
\title{\LaTeX{} to HTML Demo}
\author{Susam Pal}
\date{\normalsize 2 Sep 2020}
\begin{document}
\maketitle

\section{Euler's Identity}
In mathematics, \emph{Euler's identity} is the equality \[ e^{i \pi} + 1
= 0. \]

Euler's identity is a special case of Euler's formula from complex
analysis, which states that for any real number \( x \),
\[ e^{ix} = \cos x + i \sin x. \]

\section{Binomial Theorem}
\[ (x+y)^n = \sum_{k=0}^n {n \choose k} x^{n - k} y^k. \]

\section{Exponential Function}
\[ e^x = \lim_{n \to \infty} \left( 1+ \frac{x}{n} \right)^n. \]


\section{Cauchy-Schwarz Inequality}
\[
    \left( \sum_{k=1}^n a_k b_k \right)^2 \leq
    \left( \sum_{k=1}^n a_k^2 \right)
    \left( \sum_{k=1}^n b_k^2 \right)
\]


\section{Bayes' Theorem}
\[ P(A \mid B) = \frac{P(B \mid A) \, P(A)}{P(B)}. \]


\section{Euler's Summation Formula}

\begin{theorem}[Euler's summation formula]
If \( f \) has a continuous derivative \( f' \) on the interval \( [y,
x] \), where \( 0 < y < x \), then

\begin{align}
\sum_{y < n \le x} f(n) = & \int_y^x f(t) dt +
                            \int_y^x (t - [t]) f'(t) dt \notag \\
                          & + f(x)([x] - x) - f(y)([y] - y).
\label{theorem}
\end{align}

\end{theorem}

\begin{proof}
Let \( m = [y] \), \( k = [x] \). For integers \( n \) and \( n - 1 \)
in \( [y, x] \) we have
\begin{align*}
\int_{n-1}^n [t] f'(t) dt & = \int_{n-1}^n f'(t) dt \\
                          & = (n - 1) \bigl( f(n) - f(n - 1)
                                      \bigr) \\
                          & = \bigl(
                                n f(n) - (n - 1) f(n - 1)
                              \bigr) - f(n).
\end{align*}
Summing from \( n = m + 1 \) to \( n = k \) we find
\begin{align*}
\int_{m}^k [t] f'(t) dt & = \sum_{n = m + 1}^k \bigl(
                                n f(n) - (n - 1) f(n - 1)
                            \bigr) - \sum_{y < n \le x} f(n) \\
                        & = k f(k) - m f(m) -
                            \sum_{y < n \le x} f(n).
\end{align*}
Hence,
\begin{align}
\sum_{y < n \le x} f(n) & = - \int_{m}^k [t] f'(t) dt +
                              k f(k) - m f(m) \notag \\
                        & = - \int_{y}^x [t] f'(t) dt +
                              k f(x) - m f(y).
\label{summation}
\end{align}
Integration by parts gives us
\begin{equation*}
\int_y^x f(t) dt = x f(x) - y f(y) - \int_y^x t f'(t) dt.
\end{equation*}
When this is combined with \( \eqref{summation} \) we obtain
\( \eqref{theorem} \).
\end{proof}


\section{Hello World Program}

Here is an example of \texttt{"hello, world"} program written in the C
programming language:

\begin{verbatim}
#include <stdio.h>

int main()
{
    printf("hello, world\n");
    return 0;
}
\end{verbatim}
\end{document}
